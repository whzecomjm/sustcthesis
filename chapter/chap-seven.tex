\chapter{Examples for Cyclic Sextic Fields}
\label{chap:chap-seven}
In this chapter, we will give some examples for cyclic sextic field, using our results or methods given by chapter \ref{chap:chap-six} to a series of  problems: discriminant, integral basis, decomposition of prime, unit group and class number etc. For the first example in cyclotomic field, there exists a complete theory for that. But as a cyclic sextic number field, using the methods given by chapter \ref{chap:chap-six}, we also got the structures of it. More precisely, we discover two subfields of it, and then calculate the discriminant which is equal to the result in cyclotomic fields' theorem. Then we also give the prime decomposition of it based on our theorems in section \ref{sec:primdsex}. Also, we calculate the class number using the general theory. To solve the second field in section \ref{sec:rcsfexam}, we use almost all preliminaries we mentioned before. We first determine the Galois group of it, then we calculate the polynomial discriminant of it, based on the discriminant formula, then we find two subfields of it. After that, we also find its integral basis and get the prime decompositions. Also, we have tried to calculate the class number of it using general theory, and get the same result as M\"{a}ki did. The third section is another example given by A Bremner and B Spearman\citep{bremner2010cyclic} with sextic trinomial, but unfortunately, we haven't got the final results.

\section{7-th Cyclotomic Field}
\subsection{Cyclotomic Fields}
Let $\zeta_n$ denote a fixed primitive $n^{\text{th}}$ root of unity, and let $\mathbb{Q}(\zeta_n)$ be the number field generated by all the $n^{\text{th}}$ root of unity. The field $\mathbb{Q}(\zeta_n)$ is called the $n^{\text{th}}$ \textbf{cyclotomic field}. A following result is very import theorem.
\begin{theorem}
Let $\phi(n)$ denote the (Euler) totient of $n$, then $\mathbb{Q}(\zeta_n)$ is an Abelian extension of $\mathbb{Q}$ of degree $\phi(n)$. More precisely, there is an isomorphism:
\begin{eqnarray*}
(\mathbb{Z}/n)^{\times}&\rightarrow & \operatorname{Gal}(\mathbb{Q}(\zeta_n)/\mathbb{Q})\\
\bar{a}&\mapsto & \sigma_a,
\end{eqnarray*}
where $\sigma_a(\zeta_n)=\zeta_n^a$
\end{theorem}
Since a sub-extension of an Abelian extension is also Abelian, cyclotomic fields and their subfields already give us an abundant supply of Abelian extensions of $\mathbb{Q}$.
More formally, we have the 
\begin{theorem}[Kronecker-Weber]\label{thm:kronecker}
Every finite abelian extension of $\mathbb{Q}$ is contained in a cyclotomic field.
\end{theorem}

\subsection{7-th Cyclotomic Field}
Now let us consider the simplest cyclic sextic field, a 7-th cyclotomic field. As we all know, $K_6=\mathbb{Q}(\zeta_7)$ is generated by $\zeta_7$, which is a root of unity, and its minimal polynomial is $$f(x)=x^6+x^5+x^4+x^3+x^2+x+1.$$

First of all, we would know its structure. As we known in Section \ref{sec:unitccsf}, this field is a CM-field, i.e. a totally imaginary extension of a totally complex cyclic cubic field. Write as before, we have $K_6=\mathbb{Q}(\sqrt{m},\theta)$, where $m$ is a negative squarefree integer. Now we need to find the minimal polynomials for $\sqrt{m}$ and $\theta$.

Note that $\zeta_7^i$ is complex conjugate to $\zeta_7^{-i}$, so let $\xi=\zeta_7+\frac{1}{\zeta_7}$, then $\xi$ is a real number\footnote{One can find that $\xi=2\cos\frac{2\pi}{7}$.}. Since $\zeta_7$ is a root of the minimal polynomial, then we have $$\zeta_7+\frac{1}{\zeta_7}+\zeta_7^2+\frac{1}{\zeta_7^2}+\zeta_7^3+\frac{1}{\zeta_7^3}+1=0,$$ Substitute $\xi$ into the equation, we have \begin{equation}\label{eqn:xi}
g(\xi)=\xi^3+2\xi^2-2\xi-1=0
\end{equation}
Let $x=\xi-\frac{2}{3}$, we can change it into the standard form, i.e. $$x^3-x^2-2x+\frac{13}{27},$$
where we can get $e=7,u=-1,v=1$, hence, the cyclic cubic subfield defined by $\xi$, where $\xi$ is a solution of equation \ref{eqn:xi}. i.e. $K^+=K_3=\mathbb{Q}(\xi)$.

From theorem \ref{thm:ccpoly}, we got the discriminant of $K_3$ is $49=f_3^2$, and since $v=1$, we got that $(1,\xi,\xi^2)$ is a power integral basis of $K_3$.

On the other hand, note that the automorphism $\sigma_2:\zeta_7\rightarrow\zeta_7^2$ generates the subgroup of order 3. Thus consider $\omega=\zeta_7+\zeta_7^2+\zeta_7^4$.
From the properties of root of unity, we have following two equations: $$(\zeta_7+\zeta_7^2+ \zeta_7^4)+(\zeta_7^3+\zeta_7^6+\zeta_7^{12})=-1,$$
$$(\zeta_7+\zeta_7^2+ \zeta_7^4)(\zeta_7^3+\zeta_7^6+\zeta_7^{12})=3-1=2$$

Then $\omega$ satisfies a quadratic equation: $$h(x)=x^2+x+2=0,$$ hence $\mathbb{Q}(\omega)$ is a quadratic subfield, and $K_2=\mathbb{Q}(\sqrt{-7})$. Then the discriminant of $K_2$ is $-7=f_2$. From theorem \ref{thm:sexdisccom}, we have the discriminant of $K_6$ is $d(K_6)=(-7)\times(7)^2\times(7)^2=-7^5$.

A famous result for discriminant of $\mathbb{Q}(\zeta_p)$, where $p$ is a prime, is that $\operatorname{Disc}(\mathbb{Q}(\zeta_p))=(-1)^{(p-1)/2}p^{p-2}$. In this field, we have $\operatorname{Disc}(\mathbb{Q}(\zeta_7))=(-1)^{3}7^{7-2}=-7^5$, which coincides our result.

To sum up, we got that $K_6=\mathbb{Q}(\xi,\sqrt{-7})$. As for the prime decomposition, the discriminant has only one prime divisor, namely $7$, hence only $7$ is ramified in $K_6$. what's more, since $7$ is common factor of $f_2$ and $f_3$, hence it is totally ramified. In fact, since $K_6$ is monogenic, we have $(x+6)^6 (\operatorname{mod} 7)$. For other primes, we just need to consider the decomposition in subfields, the using the result in section \ref{sec:primdsex}. For example, consider $p=37$, note that $$\left(\frac{-7}{37}\right)=(-1)^{(37-1)/2}\left(\frac{7}{37}\right)=(-1)^{(7-1)(37-1)/4}\left(\frac{37}{7}\right)=-\left(\frac{2}{7}\right)=1$$ Hence $37$ is ramified in $K_2$. On the other hand, similarly, consider we just need to verify that $49^{(37-1)/3}\equiv 1 (\operatorname{mod }37)$ or not. However, $49^{(37-1)/3}\equiv 12^{12} \equiv (-4)^6\equiv (-10)^2\equiv 26\not\equiv 1 (\operatorname{mod} 37)$, hence $37$ is inert in $K_2$. Hence, $37=\wp_1\wp_2$.

As for the unit group of $K_6$, we first refer a theorem  \citep{Xianke2006ANT} which is a corollary of theorem \ref{thm:unitcomplexsec} as follows:
\begin{theorem}
Let $m=p^s$, where $p$ is an odd prime, $s\in\mathbb{N^*}$, then $K=\mathbb{Q}(\zeta_m)$ has the same system of fundamental units to $K^+=\mathbb{Q}(\zeta_m+\zeta_m^{-1})$.  
\end{theorem}

So from Dirichrlet Theorem \ref{thm:Dirichlet}, $U(K_6)=\mu(K_6)\times \mathbb{Z}^2$, where $\mu(K_6)=\langle\zeta_7\rangle$ is the root of unity of $K$ and the fundamental units are the same to $K_3$. The fundamental units in $K_3$ are $$(-1+\xi+\xi^2,2-\xi^2)=(1+\zeta_7+\zeta_7^2+\zeta_7^{-1}+\zeta_7^{-2},-\zeta_7^{-2}-\zeta_7^{2}).$$

As for the class number, firstly, we compute the Minkowski's bound, and we get $$C(K_6)=\left(\frac{4}{\pi}\right)^3\frac{6!}{6^6}\sqrt{|-7^5|}=4.13$$
Hence $Cl(K_6)=\langle[\wp]|\wp|2\text{ or }3\rangle$

now we consider the decomposition of $2$ and $3$. since $\left(\frac{-7}{2}\right)=1$, hence it splits in $K_2$. On the other hand, $x^3+x^2-2x-1\equiv x^3+x^2+1 (\operatorname{mod} 2)$, i.e. it's inert in $K_3$, hence $2=\wp_1\wp_2$\footnote{Here we haven't use cubic reciprocity, since there is no definition for $p=2$(in fact it only define in $p=3k+1$, like quadratic reciprocity defined in $p=2k+1$).}. In fact, $\wp_1=(2,1+\xi+\xi^3),\wp_2=(2,1+\xi^2+\xi^3)$, since $1+\xi^2+\xi^3$ is conjugate to $1+\xi+\xi^3$, hence $[\wp_1]=[\wp_2]=1$.

For $p=3$, it's easy to see that $\left(\frac{-7}{3}\right)=-1$ and $x^3+x^2-2x-1$ is irreducible in $\mathbb{F}_3$, hence $p=3$ is inert in $K_6$. To sum up, we have the class number is $1$.


\section{A Real Cyclic Sextic Field}\label{sec:rcsfexam}
Consider the sextic field $K_6$ generated by a root of $$f(x)=x^6-x^5-6x^4+6x^3+8x^2-8x+1.$$ First note that $f(x)$ is irreducible over $\mathbb{Q}$, since $f(x)\equiv x^6+x^5+1 (\operatorname{mod} 2)$. In order to verify the Galois group of the splitting field of $f$, we should first compute the resolvent polynomials. Let $p_1=x_1+x_2$, $p_2=x_1+x_2+x_3$, then use the numerical method, we have the approximate root of $f$ are: $$(r_1,r_2,r_3,r_4,r_5,r_6)=(-1.97766,-1.46610,0.14946,0.73068,1.65248,1.91115)$$
Then, \begin{eqnarray*}
R_{p_1,f}(y)&=&\prod_{p_{1i}\in orb(p)}(y-p_{1i}(r_1,\dots,r_6))\\
&=& x^{15}-5x^{14}-14x^{13}+98x^{12}+7x^{11}-567x^{10}+280x^9+1404x^8\\
&&-818x^7-1596x^6+735x^5+700x^4-203x^3-77x^2+11x+1\\
&=&(x^3-x^2-2 x+1)(x^6-2 x^5-10 x^4+6 x^3+30 x^2+17 x+1)\\
&&(x^6-2 x^5-10 x^4+27 x^3-12 x^2-4 x+1)
\end{eqnarray*} 

So we have $(3,6^2)$ cycle, from table \ref{tab:allorbits}, we have that the Galois group of $f$ is either $C_6$ or $D_6$. Then we calculate $R_{p_2,f}$,
\begin{eqnarray*}
R_{p_2,f}(y)&=&\prod_{p_{2i}\in orb(p)}(y-p_{2i}(r_1,\dots,r_6))\\
&=& x^{20} + 2 x^{19} - 106 x^{18} - 600 x^{17} - 593 x^{16} + 3252 x^{15} - 6530 x^{14}\\
&& -2589 x^{13} + 4875 x^{12} - 675 x^{11} - 1759 x^{10} + 3349 x^9 + 5376 x^8 + 1260 x^7 \\
&&+ 1188 x^6 + 865 x^5 + 316 x^4 + 77 x^3 + 23 x^2 + 11 x + 1
\\
&=&(x^2 - x + 1) (x^6 - 12 x^5 + 3 x^4 + 7 x^3 + 6 x^2 - 2 x + 1) \\
&&(x^6 + 8 x^5 + 25 x^4 + 2 x^3 + 5 x^2 + 2 x + 1) \\
&&(x^6 + 7 x^5 - 8 x^4 + 4 x^3 + 27 x^2 + 12 x + 1)
\end{eqnarray*}  
So we have $(2,6^3)$ cycle, from table \ref{tab:allorbits}, we have that the Galois group of $f$ is $C_6$, also we know $K_6$ is totally real.

The polynomial discriminant of $f$ could be given by formula \ref{prop:disctri}, in fact, $\operatorname{Disc}(f)=453789=3^3\times7^5$
Note that, $d(K_6)=f_2f_3^2f_6^2$ by theorem \ref{thm:sexdisc}, since $f_6=\operatorname{lcm}(f_2,f_3)$, hence we have $f_2^3|d(K_6)$ and $f_3^4|d(K_6)$. The relation between $d(K_6)$ and $\operatorname{Disc}(f)$ is showed in lemma \ref{lem:discpoly-field}, i.e. $\operatorname{Disc}(f)=a^2d(K_6)$, here we have $$3^3\times 7^5=a^2 f_2f_3^2f_6^2,$$ from the requirements we mentioned above, we have $$a^2f_3^4|3^3\times7^5,$$ this force $f_3=7$, and $a=1$ or $a=3$. Then $a^2f_2^3f_3^2|3^3\times7^5$, i.e. $$a^2f_2^3|3^3\times7^3.$$ If $a=3$, then $f_2=7$, and for this case $d(K_6)=7^5$, and we get $3^3d(K_6)=\operatorname{Disc}(f)$. An contradiction! Hence, $a=1$, and we could get $f_2=3\times7$.

To sum up, we have $f_2=21,f_3=7$. So we can immediately get the quadratic subfield is $\mathbb{Q}(\sqrt{21})$, with minimal polynomial $g(x)=x^2-x-5$, and integer ring $\mathbb{Z}\left[\frac{1+\sqrt{21}}{2}\right]$. \footnote{In fact, since $\phi(21)=\phi(3)\times\phi(7)=12$ and $6|12$, actually this cyclic sextic field is a subfield of the cyclotomic field $\mathbb{Q}(\zeta_21)$.}  As for the cyclic cubic subfield, since $e=f_3=7$, and $v=1$, hence we have the minimal polynomial of $\theta$, $h(x)=x^3-x^2-2x+\frac{13}{27}$ with power integral $(1,\theta,\theta^2)$. To sum up, $$K_6=\mathbb{Q}(\theta,\sqrt{21})$$

Since $d(K_6)=\operatorname{Disc}(f)$, hence $K_6$ has a power integral basis, wlog, let $r$ be a root of $f$, then $(1,r,r^2,\dots,r^5)$ is a integral basis. On the other hand, we have another integral basis given by theorem \ref{thm:intbasis_sex}.

The prime decomposition is quite easy, $7$ is totally ramified in $K_6$, since it is ramified in both two subfield. In fact, $f(x)\equiv (x+1)^6 (\operatorname{mod} 7)$. Hence $7O_{K_6}=\wp^6$, where $\wp=(7,r+1)$. $3$ is ramified in $K_2$, and inert in $K_3$, hence $2O_{K_6}=\wp'^2$. The other prime's situation could be solved through theorem \ref{thm:csf-unramified}.

As for the class number, we first compute the Minkowski's bound, $$C_{K_{6}}=(4/\pi)^{0}\frac{6!}{6^6}\sqrt{453789}=10.4,$$ now we should consider $p=2,3,5,7$. we can find that $2$ is inert in $O_{K_6}$. $7$ is totally ramified with $\wp=(7,r+1)$, while $N(r+1)=7$\footnote{calculate through $f(x-1)$}, $(r+1)\subset(7)$, hence $\wp=(r+1)$ is a principal ideal.

As for $p=3$, we have $f(x)\equiv(2+x+x^2+x^3)^2 (\operatorname{mod} 3)$, hence $\wp'=(3,2+r+r^2+r^3)$, $N(2+r+r^2+r^3)=3$ hence $\wp'=(2+r+r^2+r^3)$. Similar result for $p=5$. Finally, we get $h(K_6)=1$.

\section{Sextic Trinomials}
As expected, increasing the degree of a polynomial will increase its complexity. In the case of sextics, however, the number of possible Galois groups jumps up to sixteen. We can once again find a list of these groups in Cohen's book and we expect $S_6$ to be the most frequently occurring group. 

It is much more preferable to work with sextic trinomials rather than general sextics. Again, we can reduce the possible unique forms of these trinomials to only $x^6+ax+b,x^6+ax^2+b,x^6+ax^3+b$. Note that the last of these three forms can be simplified to a quadratic in $x^3$.

It has already been shown by A Bremner and B Spearman \citep{bremner2010cyclic} that up to scaling, there exists a single, unique sextic trinomial with Galois group isomorphic to $C_6$; which was given as
$$f(x)=x^6+133x+209$$

Now we focus on this example, i.e. $K_6$ is the splitting field of $f(x)$. First of all, since $f(x)\equiv x^6+x+1(\operatorname{mod} 2)$, hence $f(x)$ is irreducible over $\mathbb{Q}$.  Then, we can compute the discriminant of the polynomial, from formula \ref{for:nxpxq}, we have $$\operatorname{Disc}(f)=(-1)^{\frac{4\times5}{2}}\cdot 5^5\cdot 133^6+(-1)^{\frac{6\times5}{2}}\cdot 6^6\cdot 209^5=-19^5\times83^2\times277^2.$$

We have $d(K_6)=f_2f_3^2f_6^2$ by theorem \ref{thm:sexdisc}, since $f_6=\operatorname{lcm}(f_2,f_3)$, hence we have $f_2^3|d(K_6)$ and $f_3^4|d(K_6)$. The relation between $d(K_6)$ and $\operatorname{Disc}(f)$ is showed in lemma \ref{lem:discpoly-field}, i.e. $\operatorname{Disc}(f)=a^2d(K_6)$. Whatever, $f_3^4|d(K_6)$ force that $f_3=19$ and $-19|f_2$. There are several possible cases for $d(K_6)$:
\begin{equation*}
\left\{ \begin{array}{l}
f_3=19,f_2=-19;\\
f_3=19,f_2=-19\times 83;\\
f_3=19,f_2=-19\times 277;\\
f_3=19,f_2=-19\times 83\times 277.
\end{array} \right.
\end{equation*}
To recognize the case for the field generated by a root of $f$, we need another method. However, this example is still under solving.

