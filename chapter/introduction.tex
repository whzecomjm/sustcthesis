\chapter{Introduction}
\label{chp:introduction}
Formally, there are 2 parts and 8 chapters in this thesis. The first 4 chapters, namely chapter 2-4, mainly refer to some reviews for Galois theory and Algebraic Number theory etc. In chapter 6-8, we obtain some results on cyclic sextic fields and Galois extensions of function fields. And then, we also give some examples using our consequences. 

Algebraic Number Theory and Algebraic Geometry are two central researching frontiers of pure mathematics. To find the discriminant, integral basis, class number of a algebraic number field is a classical question in algebraic number theory. More accurately, if we consider the splitting field of the polynomials over $\mathbb{Q}$, which is certainly Galois, then some restrictions from Galois extension will affect the structures of the field. Igor Shafarevich showed that every finite solvable group $G$ is realizable over $\mathbb{Q}$, i.e. there exists a field $K$ such that $\operatorname{Gal}(K/\mathbb{Q})\cong G$. This is related to a famous question, the Inverse Galois Problem. For our cases, we focus on some simple cyclic fields which have cyclic group as its Galois group over $\mathbb{Q}$. 

We introduce Galois theory in chapter \ref{chap:chap-three}. The Galois theory is the principal connection between the two parts of the thesis. In chapter \ref{chap:chap-three}, we will first introduce some necessary preliminaries of Galois Theory for our results based on Emil Artin's book \citep{artin1944galois} and David A. Cox's book\citep{cox2012galois}. Then, we focus on computing the Galois group of a polynomial. We sort out some useful materials from A. Healy \citep{healy2002resultants}, C. Bright \citep{bright2013computing}, and L. Soicher and J. McKay \citep{soicher1985computing}.

One will find that chapter 3-5 have almost the same content's structure, i.e. we have several common sections like the discriminant, integral basis, decomposition of prime, unit group and class number and so on. In fact, Chapter 3 is the reviews of general theory for number fields and some analytic number theory. We have paid much time on sorting out the materials from books of Prof. Xianke Zhang \citep{Xianke2006ANT}, Henri Cohen \citep{cohen1993course} and Richard Molin \citep{mollin1999algebraic}. In chapter 4 and 5, we have clearly rearranged those results for quadratic field and cyclic cubic field based on Cohen and Zhang's books. Some of results has been modified for the simplicity. In chapter 5, we have posed an example for prime decomposition in cyclic cubic field and two examples for computing the class number of those fields. What's more, we have recovered the results of Seidelmann's paper \citep{seidelmann1917gesamtheit} through a lemma in Cohen's book. We also give a formula (See theorem \ref{thm:ccuformula}) for a type of cyclic cubic field which is given risen by the ideas of F.C. Orvay's \citep{orvay1991cyclic}.

In chapter 6, we first refer to Sirpa M{\"a}ki's results \citep{maki1980determination} on discriminant and conductor of a cyclic sextic field, and then get the integral basis and prime decomposition of it.  M{\"a}ki give a direct result for discriminant, we have tried to recover the result to get it using the conductor-discriminant formula. Then with this message, we complete to find the integral basis of cyclic sextic field. At section \ref{sec:primdsex}, we succeed in obtaining all the cases for decomposition of prime numbers. For real cyclic cubic field, the unit group and class number has been solved for some ``small" (actually is quite large) discriminants, we can find a table in M{\"a}ki's book. For complex cyclic sextic field, it's a CM-field, hence the class number and the unit group could be reduced into its cyclic cubic field. But unfortunately, we haven't got the final precise results of them.

In chapter 7, we give two examples. One of them is a complex cyclic sextic field and the other is a real cyclic sextic field. For this a complex cyclic sextic field, i.e. 7-th cyclotomic field, we use our results in chapter 6 to get its discriminant and the prime decomposition, integral basis. Then we verify those results through the theory of  
cyclotomic field. Since this field is quite simple, we use the general theory of class number to find its class number is 1. Then, we also given an inconspicuous example. We first verify the Galois group of the polynomial using the method in chapter 2, then we first compute the polynomial discriminant. After that, using the discriminant formula, we confirm the subfields of it, and also do other processes as we mentioned above except the unit group. We also find another example proposed by A Bremner and B Spearman\citep{bremner2010cyclic} with sextic trinomial, but unfortunately, we haven't got the final results.

Although there is only one chapter for part two, i.e. the Galois extensions in a function field, this is still an important part of my thesis. We have proved two theorem for specializing L\"{u}roth's theorem within Galois extension. More precisely, assuming that $K(x)/E$ is Galois, without using L\"{u}roth's theorem, we prove that $E=F(u)$, where $u\in K(x)$ can be determined by the elementary symmetric polynomials. We then give a classical example for $\operatorname{Gal}(K(x)/E)=D_3$. 


